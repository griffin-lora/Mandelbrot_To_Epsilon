\documentclass[12pt]{article}
\usepackage{graphicx} % Required for inserting images
\usepackage[margin=1in]{geometry}
\usepackage{fancyhdr}
\usepackage{graphicx}
\usepackage{enumitem}
\usepackage{amsmath}
\usepackage{amsfonts}
\usepackage{amsthm}
\usepackage{enumitem}
\everymath{\displaystyle}
\rfoot{\thepage}
\setlength{\headheight}{15pt}

\theoremstyle{definition}
\newtheorem{definition}{Definition}

\theoremstyle{definition}
\newtheorem{theorem}{Theorem}

\theoremstyle{definition}
\newtheorem{lemma}{Lemma}

\theoremstyle{definition}
\newtheorem{example}{Example}

\begin{document}
\begin{flushleft}


    
\begin{definition}[Epsilon Function]
    Let $S \subseteq \mathbb{C}$, $f\colon S\to\mathbb{R}$, and $\varepsilon > 0$, then an \textit{estimate function} of $f$ is a function $f_\varepsilon\colon S\to\mathbb{R}$ where for all $z \in S$ that $\left|f_\varepsilon(z) - f(z)\right| < \varepsilon$.
\end{definition}

\begin{definition}[Minimum Distance Function]
    Let $S \subseteq \mathbb{C}$, then $\mu\colon\mathbb{C}\to\mathbb{R}$ as $\mu(c) = \inf\{ |z - c| : z \in S \}$ is the \textit{minimum distance function} for S
\end{definition}

\begin{definition}[Distance Epsilon/Estimate Function]
    If $\mu$ is a minimum distance function for $S$ then $\mu_\varepsilon$ is a \textit{distance epsilon function} or \textit{distance estimate function} for $S$.
\end{definition}

\begin{lemma}
    $\mu_\varepsilon(c) \geq 0$
\end{lemma}

\begin{lemma}
    If $\mu_\varepsilon(c) = 0$, then there exists some $z \in \mathbb{C}$ where $|z - c| < \varepsilon$ such that $z \in S$.
\end{lemma}
\begin{proof}
    First consider that
    \begin{align*}
        |\mu_\varepsilon(c) - \mu(c)| = |\mu(c)| = \mu(c) < \varepsilon
    \end{align*}
    Now consider $z \in S$ such that $|z - c| = \mu(c)$, then $|z - c| = \mu(c) < \varepsilon$.
\end{proof}

\begin{theorem}
    If $\mu_\varepsilon(c) < \varepsilon$, then there exists some $z \in \mathbb{C}$ where $|z - c| < \varepsilon$ such that $z \in S$.
\end{theorem}

% \begin{definition}[Mandelbrot Set]
%     Let $f_c\colon\mathbb{C}\to\mathbb{C}$ as $f_c(z) = z^2 + c$ and let $f_c^n$ denote $n$ iterations of $f_c$, then the Mandelbrot set $\mathcal{M} = \{ c \in \mathbb{C} : \lim_{n\to\infty} f_c^n(0) \text{ converges} \}$
% \end{definition}

\end{flushleft}
\end{document}